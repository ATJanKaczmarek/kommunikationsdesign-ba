\documentclass[
    a4paper,
    11pt,
    bibliography=totoc,
    listof=totoc,
    parskip=half,
    headings=normal
]{scrbook}

% ------------------------------------------------------------------------------
% Packages
% ------------------------------------------------------------------------------
\usepackage[utf8]{inputenc}
\usepackage[T1]{fontenc}
\usepackage[ngerman]{babel}     % German language support

\usepackage[autostyle=true]{csquotes} % Context sensitive quotes
\usepackage{lmodern}            % Modern font
\usepackage{graphicx}           % Graphics support
\usepackage{xcolor}             % Color support
\usepackage{scrlayer-scrpage}   % Header and footer control

% Bibliography setup
\usepackage[
    backend=biber,
    style=apa,
    sorting=nyt
]{biblatex}
\addbibresource{main.bib}

% Hyperlinks (should be loaded last)
\usepackage[
    hidelinks,
    breaklinks=true,
    pdfauthor={Jan Kaczmarek},
    pdftitle={Designgeschichte und -theorie},
    pdfsubject={Kommunikationsdesign}
]{hyperref}

% ------------------------------------------------------------------------------
% Document Start
% ------------------------------------------------------------------------------
\begin{document}

\frontmatter

% Title Page
\begin{titlepage}
    \centering
    \vspace*{2cm}
    {\scshape\LARGE Designgeschichte und -theorie\par}
    \vspace{1.5cm}
    {\huge\bfseries Von der Antike bis zur Neuzeit\par}
    \vspace{2cm}
    {\Large\itshape Jan Kaczmarek\par}
    \vfill
    Studiengang Kommunikationsdesign\par
    \today\par
\end{titlepage}

\tableofcontents

\mainmatter

% ------------------------------------------------------------------------------
% Content
% ------------------------------------------------------------------------------

\chapter{Einleitung}
Betrachtet man die Antike aus der Gegenwart, fällt der erste Gedanke auf das \textit{Imperium Romanum}. Ein zweiter Gedanke richtet sich häufig auf das antike Griechenland. Dabei ist die Antike keine zweiteilige Epoche. Sie brachte nicht die ersten Hochkulturen hervor, auch wenn es aus einer eurozentrischen Perspektive so wirken mag. Dennoch ist die Antike eine Epoche, die enormen Einfluss auf die zivilisatorische Entwicklung genommen hat. Die Antike untergliedert sich historisch in 6 weitere Epochen. Auch wenn in dieser Arbeit der Fokus auf der Designgeschichte liegt, muss im Rahmen der zeitlichen Einordnung zunächst etwas tiefer in die Geschichte eingetaucht werden, damit gestalterische Strömungen erschlossen werden können.

\section{Zeitliche Einordnung der Antike}
Am Strahl der Zeit betrachtet beginnt die Antike circa 800 v. Chr. in Griechenland und endet um 400 n. Chr. mit der Teilung und dem darauffolgenden Untergang des römischen Reiches. Diese in etwa 1200 Jahre umspannende Epoche fasst sechs Unterepochen zusammen. Beginnend mit der Archaik, die die erste bekannte Form von Demokratie hervorbrachte. Die griechischen Stadtstaaten (Poleis)  gingen aus prosperierendem Handel hervor und währten 300 Jahre.

\section{Ursprung}
Um 500 v. Chr. beginnt in Griechenland die Klassik, zeitgleich wird in Rom die Republik ausgerufen, nachdem die Stadt zuvor von einem König regiert wurde. Griechenland und Rom erweiterten in diesem Zeitraum ihr Einflussgebiet stark. Während die prägende Klassik nach bereits 170 Jahren im \textit{Hellenismus} (Zeit unter Alexander des Großen) mündete und unter dessen Einfluss weiter expandierte, ging das antike Griechenland mit dem Beginn der Kaiserzeit ins \textit{Imperium Romanum} über. Das riesige Einflussgebiet erstreckte sich von Britannien bis Nordafrika und reichte von der Westküste des heutigen Portugals bis in den arabischen Raum. Die über 400 Jahre andauernde Expansion endete um 400 n. Chr. in der Teilung in das Weströmische und Oströmische Reich.

Abbildung 1: Zeitstrahl

\section{Griechische Antike}
Die griechische Hälfte der Antike -- besonders herauszustellen ist dabei die Klassik -- brachte zahlreiche Errungenschaften in Philosophie, Dichtkunst (Entstehung des Dramas) und der Mathematik hervor. Allesamt beeinflussten die Kultur und das Zusammenleben in Städten und nahmen dadurch indirekt oder direkt Einfluss auf die Architektur \autocite[14]{hodgeKurzeGeschichteArchitektur2022}. Die mathematischen Erkenntnisse dieser Zeit können als kritische Bedingungen für die Bauweise gesehen werden. Besonders die auf Euklid zurückgehenden geometrischen Berechnungsmethoden ermöglichten erst Längen- und Maßrechnungen sowie die für die antike Architektur substanzielle Proportionslehre.\newline
Man darf aufgrund der vielen “plötzlichen” Erkenntnisse nicht davon ausgehen, dass das antike Griechenland aus dem Nichts entstand. Bereits zuvor gab es im östlichen Mittelmeerraum diverse Kulturen, die regen Handel trieben, lokale Ausprägungen von Kulturen und Religionen pflegten und die Baukunst vorantrieben. Die Jahrhunderte vor dem Aufstreben der griechischen Antike werden als Dunkle Jahrhunderte bezeichnet, da nur wenige Überreste aus dieser Zeit vorhanden sind und eine historische Rekonstruktion von 1200 v. Chr bis 800 v. Chr nur schwer bis nicht möglich ist. Belegt ist jedoch, dass die Architektur darauffolgend dabei über Jahrhunderte weiterentwickelt wurde, sodass im 4. Jahrhundert v. Chr. ein Regelsystem existiere, das für alle Bauwerke Vorgaben machte \autocite[22]{hodgeKurzeGeschichteArchitektur2022}. Daraus lässt sich schließen, dass eine Konzentration der kulturellen und religiösen Strömungen für eine kollektive Entwicklung neuer Konzepte sorgte. Zukowsky schreibt dazu: \enquote{Die Biegebauweise geht auf vorgeschichtliche Zeiten zurück z. B. Stonehenge (ca. 3000-2000v. Chr.). Die griechische Architektur treibt diese \enquote{Grundlage} auf den Höhepunkt.} \autocite[22]{zukowskyGeschichteArchitekturPyramide2022}. Mit Biegebauweise ist dabei vmtl. der Rundbau gemeint, der zentral in der Baukunst der gesamten Antike ist. Mithilfe von Säulen werden kreisförmige Grundrisse ermöglicht. Der römische Architekt \textit{Vitruvius} schrieb in seinem Werk \textit{De Arcitectura} (dt. Über Architektur) Regeln und Gesetze für Bauwerke nieder. Als einziges aus der Antike überliefertes Buch über Architektur hat es einen enormen Stellenwert in der Historik. \textit{Vitruvius} klassifiziert die verwendeten Säulen in die dorische Ordnung, die ionische Ordnung und die korinthische Ordnung (drei Hauptordnungen). Dabei unterscheiden sich die Säulentypen in ihrer Beschaffenheit als auch in ihren geografischen Ausprägungen. Die regionalen Unterschiede der Säulen können als Beleg verstanden werden, dass es sich beim antiken Griechenland keinesfalls um eine homogene Bevölkerung handelte. Gleichzeitig belegen die überregionalen Gemeinsamkeiten der Säulen, dass es im antiken Griechenland feste Regeln gab, die ubiquitär zum Einsatz kamen.

\paragraph{Exkurs: Euergetismus}
Im \textit{Hellenismus} vergrößerte sich das griechische Einflussgebiet stark. Neue Gebiete wurden oder waren bereits erschlossen. Dabei geschah dies nicht immer durch Annexionen, sondern häufig auch durch Euergetismus wobei das (durch Plünderungen) reiche Griechenland prunkvolle Gebäude und Kulturereignisse wie Theaterbauten und Aufführungen stiftete. Die Schenkungen waren dabei stets politisch motiviert und dienten dazu, die griechische Kultur ins Zentrum der Menschen dieser Regionen zu rücken. Die Hellenisierung der umliegenden Völker und deren Aufgang im griechischen Territorium hatte ebenfalls einen reziproken Einfluss auf die -- sowieso heterogene -- “griechische” Kultur.

An dieser Stelle ist es erwähnenswert, dass dem antiken Griechenland die mykenische und minoische Kultur vorausging und enorme Einflüsse auf Kunst und Glauben (Vasenmalerei) hatte. Auch das Alte Ägypten hatte architektonische und kulturelle Einflüsse. Makedonien verstand sich zwar als Teil Griechenlands und dennoch als eigenständig. Sämtliche Strömungen mit einzubeziehen und zurückzuverfolgen, würde den Rahmen dieser Arbeit sprengen. Es zeigt jedoch gut, dass die Entwicklung der Kultur und Architektur in der Antike stets fließende Übergänge, regionale Ausprägungen und verschwimmende Grenzen hatte.

\section{Römische Antike}
Nicht zuletzt durch den Übergang Griechenlands in das \textit{Imperium Romanum} wurden die Erkenntnisse der Architektur gebündelt und von \textit{Vitruv}ius einem Baumeister in Rom, in der zweiten, römischen Hälfte der Antike als Regeln in sein Buch übertragen. “Zu den vielen römischen Errungenschaften zählen Kuppeln, die für Tempel, Bäder, Villen, Paläste und Gräber erbaut [...] wurden.”, zudem ist der steinerne Rundbogen ein Produkt der römischen Architektur, der den Weg für Aquädukte und fortschrittliche Bauten wie das Kolosseum ebnete \autocite[16]{hodgeKurzeGeschichteArchitektur2022}. Daneben bedienten sich die antiken Römer neuer Materialien, wie Beton (und entwickelten die notwendige Schalenbauweise) oder Glas \autocites[16,208]{hodgeKurzeGeschichteArchitektur2022}[28]{zukowskyGeschichteArchitekturPyramide2022}.Damit baut die römische Architektur auf der griechischen auf und weitet die Baukunst mit neuem Wissen und Errungenschaften enorm aus.

Mehrere Millionen Quadratkilometer umfasste das römische Einzugsgebiet zu Hochzeiten. Schätzungen über die damalige Population gehen von 50 bis 65 Millionen Menschen aus \autocite[813--814]{frierDemography1998}. Die Eroberung neuer Gebiete war der erste Schritt zur Vergrößerung des \textit{Imperium Romanum}; darauf folgte eine radikale Politik, die die fremden Kulturen römisch machen sollte.

\chapter{Produkte im römischen Reich}
\enquote{Wo immer die Römer neue Gebiete eroberten, errichteten sie neue Städte.} \autocite{hodgeKurzeGeschichteArchitektur2022}. Und wo immer die Römer neue Städte errichteten, entstanden neue Tempel, Theater, Foren und weitere Orte, die das Römischsein, die römische Religion und die römische Kultur, transportierten. Das radikale Aufoktroyieren der römischen Kultur konnte jedoch nur gelingen, wenn im ersten Schritt die Soldaten an die Außengrenzen gelangten, die (Bau-)Materialien für neue Städte durch das gesamte Reich transportiert und die Bürger:innen zweiter Klasse mit römischen Waren versorgt werden konnten. Betrachtet man die Überreste vom Pont du Gard, der Via Appia, den Horrea in Ostia Antica oder der Cloaca Maxima fällt auf, dass die römische Infrastruktur keine organisch gewachsene Sache ist, sondern, dass es sich dabei um ein strategisch geplantes System handelt. Befestigte öffentliche Straßen (Viae Publicae), die eine effiziente Fortbewegung und eine rasche Übermittlung von Nachrichten ermöglichten, waren zentral dafür \autocite{thieleAlleWegeFuehren2010}. Daneben dienten die Land- und Wasserwege zum Transport von Gütern, die im gesamten römischen Reich produziert wurden und quer durch das Imperium verfrachtet wurden.

\subsection{Amphoren}
Ein standardisiertes Hohlmaß -- die Römische Amphore -- ermöglichte einen effizienten Transport von Flüssigkeiten und Schüttgütern wie Ölen, Wein und Mehl.
Das Corpus Inscriptionum Latinarum (kurz: CIL) ist eine systematische Sammlung antiker lateinischer Inschriften \autocite{CorpusInscriptionumLatinarum2025}. Der Forscher Heinrich Dressel bereiste für die Erweiterung des CIL in den Jahren 1874--1876 und 1878 Italien und stellte ausführliche Untersuchungen zu Inschriften an. Dabei beschäftigte sich der Epigraphiker intensiv mit Scherbenfunden von Amphoren und stellte eine systematische Typentafel auf, die die Grundlage für weitere Forschung auf diesem Gebiet liefert, welche sich mit den römischen Handelswegen, den Verwendungszwecken der Amphoren sowie deren Verbreitung auseinandersetzt. \autocites{schallmayerWegmarkenAntikenWelthandels2014}{ehmigAhnlichesIstNicht2004}{ehmigEpigraphischeZeitreisenHeinrich2025}.

Abbildung 2: Amphorentypen nach Dressel

Die Behältnisse waren teilweise stapelbar und konnten übereinander möglichst effizient in die Laderäume von Schiffen gelagert werden. Die Besonderheit der Gefäße steckt in deren Standardisierung. Technische Spezifikationen bestimmten, wie eine Amphore getöpfert werden musste. In den Produktionsstätten gab es Standardformen, die der Massenproduktion dienten. So vereinfachten sie den Handel und das Quantifizieren von Beständen (Amphore als Maßeinheit) \autocite{gonzalezcesterosISOContainerAntikeStandardisierte2023}.

\subsection{Terra Sigillata}
Töpferware war im römischen Imperium ein Massenprodukt. Das beliebte und weit verbreitete Tafelgeschirr der \textit{Terra Sigillata} zeichnet sich durch die glänzende rote Farbe und die feinen Verzierungen aus. Diese Feinkeramik wurde in hoher Stückzahl angefertigt. Ermöglicht wurde dies durch die Verwendung von Formschlüsseln, die den Ton in Form zogen. Mithilfe von Punzen wurden die Motive und Verzierungen auf die Töpferware gebracht.

Die einzelnen Produktionsschritte wurden arbeitsteilig erledigt.


\subsection{Mögliche weitere Massenprodukte}
Die Vorteile einer standardisierten Produktion liegen auf der Hand:

Die Produkte lassen sich in \textbf{großen Auflagen}, \textbf{zu günstigen Preisen} und \textbf{verhältnismäßig schnell} produzieren. Je simpler (unkomplexer) ein Produkt gestaltet ist bzw. je weniger Arbeitsschritte für dessen Herstellung notwendig sind, desto effizienter (gesteigerte Produktion in gleicher Zeit) wird die Produktion.

Um, wie in der Aufgabenstellung dieser Arbeit gewünscht, Vermutungen anzustellen, welche weiteren Produkte (neben den bereits genannten Amphoren und der TS) massenhaft hergestellt wurden, werden nun drei Vermutungen angestellt, welche im Anschluss genauer betrachtet werden.

\subsubsection{Öl, Wein, Mehl}
Die Bevölkerung des römischen Reiches musste verpflegt werden. Vor dem Hintergrund der bereits angeführten Zahl der Einwohner:innen des Reiches, stellt sich die Frage, wie die Verpflegung des Volkes funktionierte. Klar ist jedoch, dass massenhaft Nahrung hergestellt werden musste. Da das Thema \textit{Verpflegung} eine eigene Arbeit füllen könnte, kann nicht vollumfänglich auf die Herstellung von Nahrungsmitteln eingegangen werden. Dennnoch soll versucht werden einen Einblick in die Nahrungsmittelproduktion zu gewinnen.

Die geänderten Anforderungen (Bevölerkungswachstum, Gebietserweiterung) im Laufe der Zeit haben zu Änderungen in der Verpflegungsproduktion geführt. Beschränkt man den Blick lediglich auf das heutige Italien, kann man feststellen, dass das Klima alleine hier, je nach Standort, enorm variiert. Im alpinen Norden herrscht ein Gebirgsklima (Schneefall, milde Sommer, starke Temperaturschwankungen). Die stark abnehmende Höhenlage, sorgt nur etwa 50--200 Kilometer weiter südlich für völlig andere klimatische Bedingungen. Die Po-Ebene (nach dem Fluss Po) ist nicht nur weitgehend flach, sondern verfügt über ein deutlich milderes Klima. Die fruchtbaren Böden der Region dienen der Landwirtschaft. Dennoch sind Temperaturen unter 0°C im Winter möglich, was eine ganzjärige Landbewirtung verhindert.
Südwärts zeigt sich ein immer mediterraneres Klima. Die Sommer werden heißer und trockener, die Winter mildern sich deutlich, sodass im Süden Italiens wenig bis gar kein Frost auftritt. \autocite[vgl. "Klima"][]{dewiki:262975683}

Olivenbäume und Reben sind Teil der natürlichen Vegetation des Mittelmeerraums (insb. Italiens), was neben zahlreichen Quellen davon zeugt, dass bereits in der Antike Oliven und Trauben kultiviert wurden. \autocite{lodoliniComparisonFrostDamages2022} zeigen, dass Olivenbäume frostempfindlich sind. Daraus lässt sich schließen, dass Oliven im Norden des röm. Reiches nicht dauerhaft angebaut werden konnten. Dennoch schreibt \textcite[718]{harrisTrade1998} \textit{\enquote{[\dots] olives were grown further north than normally they are in modern times [\dots]}}. Der Grund dafür konnte lediglich der hohe Bedarf an Oliven sein. Das Öl wurde nämlich nicht nur als Nahrungsmittel genutzt, sondern auch als Brennstoff für die Beleuchtung von Innenräumen, sowie zur Herstellung weiterer Produkte, wie Kosmetika oder Seife verwendet. \autocite[717]{harrisTrade1998}.

\citeauthor{harrisTrade1998} geht ebenfalls auf die Produktion von Wein bzw. dessen Handel ein. Die Herstellung von Wein konnte in fast jedem Gebiet des röm. Reiches autonom erfolgen. Jedoch war Wein ein wichtiges Handelsgut. Daneben gibt \citeauthor{harrisTrade1998} eine Schätzung von 1,5 Mio. Hektolitern pro Jahr alleine für die Stadt Rom ab. Die Zahl ordnet er direkt wie folgt ein: \textit{\enquote{[\dots] only a small proportion of this demand could be met by vineyards near the city.}}. \autocite[720]{harrisTrade1998}.

Die wichtigste Basis für die Ernähung der Menschen im röm. Reich war jedoch Getreide, welches in den meisten Fällen regional bezogen wurde. Die Ausnahme lieferten dabei große Städte -- wie immer kann Rom hier angeführt werden -- welche in der Antike stets nahe Flüssen oder dem Meer angesiedelt waren \autocite[152\psq]{peacockPotteryRomanWorld1982}. Die bereits angeschnittenen Lagerhäuser (\textit{horrea}) dienten vermutlich auch der Lagerung von Getreide. Funde solcher Lagerstätten erwähnt \citeauthor{harrisTrade1998} speziell für die Stadt Rom \autocite[717]{harrisTrade1998}. Es ist naheliegend, dass in den Provinzen Nordafrikas Nahrung im Überschuss erzeugt wurde. Am Nilufer konnte acht Monate lang Landwirtschaft betrieben werden. Vier Monate des Jahres wurden die Ufer geflutet (Nilschemme, \textit{Achet}), was stets fruchtbare Böden hervorbrachte. Die komplexen Hafenbauten in den nordafrikanischen Provinzen (z. B. Kothon) verfügten wahrscheinlich ebenfalls über Lagerstätten für den Weitertransport in nördliche Teile des römischen Reiches.

Daneben machte es sich der römische Staat zur Aufgabe Getreide umzuverteilen. Das wird durch die Existens des offiziellen Amtes des \textit{praefectus annonae} \autocite[vgl.][243]{eckGROWTHADMINISTRATIVEPOSTS1998} belegt. Ziel war es die Getreideversorgung der einzelnen Provinzen aufrecht zu erhalten und falls notwendig über eine Streuer auf Getreide (\textit{cura annonae}). Ziel war es Hungernöte, die sporatisch auftraten möglichst zu unterbinden \autocite[794]{frierDemography1998}.

\subsubsection{Waffen}
Das römische Reich war eine Militärische Großmacht. Wie bereits dargelegt, wurden neue Gebiete gewaltsam erobert. Neben der Erweiterung und Sicherung der Außengrenzen, kam es im Laufe der Zeit zu zahreichen Aufständen und Bürgerkriegen innerhalb des Imperiums, die ebenfalls durch Waffengewalt unterdrückt wurden. \autocite{ListRomanCivil2025}. Die Vermutung liegt also nahe, dass Waffen in Massen produziert werden mussten.

\subsubsection{Baumaterialien}
Bis heute finden sich reichlich Überreste der römischen Bauten. Das Colosseum ist bis heute das größte jemals gebaute Amphitheater der Welt \autocite{Colosseum}. Im gesamten Einzugsgebiet des römischen Reiches sind bis heute Überreste zu finden. Der Bedarf an Baumaterialien muss also groß gewesen sein. Da Bauen im römischen Reich ein staatlich kontrollierter Prozess war, deutet es darauf hin, dass sich hier ein gesamter Industriezweig entwickelt haben muss.

\section{Vergleich der römischen Warenproduktion zur heutigen Industriegesellschaft}

\section{Gemeinsamkeiten und Unterschiede}

\section{Kleiner Exkurs: spätere “Erfindung” der Manufakturen in Frankreich}

\chapter{Antike Formensprache}
Diverse spätere Epochen bedienten sich der Formensprache der Antike. Dabei waren die Motive, warum sich Architekten und \enquote{Designer} diesem Stil bedienten unterschiedlich. In diesem Kapitel soll zunächst die Formensprache der Antike bzw. das Verständnis der Epoche der Antike so dargestellt werden, dass darauffolgend die Gründe der Rezeption in der Nachantike deutlich werden.

\section{Beschreibung der antiken Formensprache}
Die Antike ist -- wenn man sie durch eine glorifizierende Brille betrachtet -- ein Zeitabschnitt, der von Ordnung, klaren Bürgerlichen Schichten, Macht und Wohlstand geprägt ist. Ein Imperium, das von einem einzigen Kaiser regiert wird, der ein siegreicher Feldherr ist. Monumentale Gebäude, prunkvolle Tempelanlagen und Statuen aus Materialien wie Marmor, Beton oder Glas.
Die Entdeckung der Proportionen, die Säulenordnungen und monumentale Bauten sind die Grundlage für die Formensprache der Antike. Ordnung, Vernunft und politische Machtansprüche ästhetisch zu legitmieren sind die Hauptmotive.

\section{Aufgreifen der antiken Formensprache in späteren Epochen}
\subsection{Renaissance}
Für diese Arbeit spielt es keine große Rolle, welches historische Ereignis als Zeitpunkt gewählt wird, um den Beginn der Renaissance zu markieren. Wichtig ist nur zu verstehen, dass zentrale Vorkommnisse die Länder und Gesellschaften Europas prägten. Christoph Columbus \enquote{entdeckt} Amerika, der Buchdruck mit beweglichen Lettern legt den Grundstein für eine völlig neue Form der Kommunikation. Während die Kirche zu dieser Zeit noch großen Einfluss hatte, kam es zur Reformation angeführt durch Martin Luther. Die Renaissance ist wörtlich die Wiedergeburt der Antike -- die Rückbessinnung auf im Mittelalter verloren gegangenes Wissen \autocite[45]{wurlStudienheftDGUT01Designgeschichte}. Sie bildet die erste Epoche der Neuzeit. Ihre Wurzeln liegen im 14. Jahrhundert wo die geistige Strömung des \enquote{\textit{Renaissance-Humanismus}} um \enquote{\textit{Francesco Petrarca}} ein Umdenken und eine Rückbessinnung auf die Antike forderte. Petrarca studierte (genau wie seine Mitgründer der Bewegung) Vitruv's Werk \autocite[Ciapponi, 1960, zitiert nach][70]{guentherDeutscheArchitekturtheorieZwischen1988}.
\enquote{Über das ganze Mittelalter hinweg blieb Vitrv bekannt} und auch in der Renaissance \enquote{[\dots] setzten vielfältige Vitruv-Studien [in ganz Europa] ein} \autocite[69\psq]{guentherDeutscheArchitekturtheorieZwischen1988}.
Die Werke des weltberühmten Künstlers \textit{Raffael} idealisieren und verfälschen die Welt, sodass die Illusion einer perfekten Natur entsteht \autocite[49]{wurlStudienheftDGUT01Designgeschichte}. Die realistisch anmutenden Bilder übernehmen die strenge Proportionslehre und Symmetrie aus der Antiken Formensprache. Damit stellt sich der Künstler gegen die Bestrebungen der bürgerlichen Schicht der Gesellschaft, welche noch stark feudalistisch geprägt war. Maler verdienten ihr Geld meist mit Auftragsarbeiten des Adels oder der Kirche. Gleichzeitig entstand eine neue Klasse: Die \textit{Patrizier}. Diese wurde durch den Handel mit wertvollen Waren wie Textilien, Gewürzen oder Rohstoffen reich und gab ebenfalls Portraitbilder in Auftrag (vgl. Hans Holbein der Jüngere: Bildnis des Danziger Hansekaufmanns Georg Gisze in London). \textit{Hans Holbein der Jüngere} bildet im Gegensatz zu Raffael realistisch ab \autocite[45\psq]{wurlStudienheftDGUT01Designgeschichte}.

\subsection{Klassizismus}

\subsection{Historismus}

\section{Beispiele}

\section{Gründe}

\section{Kontextuelle Einordnung}

\subsection{zeitlich}

\subsection{weltpolitisch}

\subsection{gesellschaftlich}

\section{Warum waren die Formen der Antike für die Menschen späterer Jahrhunderte gestalterisch interessant?}

\section{Funktionalismus (Geschichtliche Entwicklung)}

\section{Designtheoriebegriff: “Funktionalismus”}

\section{… nicht zu tief eintauchen (später wird noch darauf eingegangen)}

\section{Frage: “Was ist funktionelles Design?” knapp beantworten}

\section{Entwicklung des Funktionalismus in der Antike}

\subsection{Vitruv}

\subsection{Entwicklung der Funktionalismus bis in die Gegenwart}

\subsection{Loos}

\subsection{Sullivan}

\subsection{Bauhaus}

\subsection{Offenbacher Ansatz}

\chapter{Funktionalismus}

\section{Funktionalismus (Auseinandersetzung)}

\section{Definition “Funktionalismus” im Design}

\section{Diskussion: “Ist Funktionalismus lediglich eine von vielen stilistischen Ausprägungen im Design? Ist Funktionalismus für Design kostitutiv?”}

\section{Verschiedene Deutungen von Design}

\section{Design ist keine Kunst}
Muss Design zweckgebunden sein?
Stellungnahme und begründete Position zur Aussage: “Design ist ohne Funktionalismus nicht denkbar.”


\backmatter

% ------------------------------------------------------------------------------
% Bibliography
% ------------------------------------------------------------------------------
\nocite{*}
\printbibliography

\end{document}
